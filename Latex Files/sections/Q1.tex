\documentclass[../main.tex]{subfiles}

\begin{document}
    \begin{flushleft}
        Given a system of ODE as follows.
        
        \begin{align}
            & \frac{dS}{dt} = bS - \beta IS \\
            & \frac{dI}{dt} = \beta IS - kI
        \end{align}
        
        It could be defined two functions, $f(S, I) = bS - \beta IS$ and $g(S, I) = \beta IS - kI$, such that
        
        \begin{align}
            & \frac{dS}{dt} = f(S, I) \\
            & \frac{dI}{dt} = g(S, I)
        \end{align}
        
        By definition, the equilibrium points are point $(S, I)$ such that $f(S, I) = g(S, I) = 0$. $(S, I) = (0, 0)$ obviously satisfies the condition and therefore an equilibrium point. In order to find the other one, consider when both $(S, I) = (0, 0)$. Hence, it is equivalent to solving the following system of equations.
        
        \begin{align}
            & (b - \beta I)S = 0 \\
            & (\beta S - k)I = 0
        \end{align}
        
        which is equivalent to solving
        
        \begin{align*}
            & b - \beta I = 0 \\
            & \beta S - k = 0
        \end{align*}
        
        Hence, the other equilibrium point is $(S, I) = (\frac{k}{\beta}, \frac{b}{\beta})$. Thus, there are two equilibrium points, $(0, 0)$ and $(\frac{k}{\beta}, \frac{b}{\beta})$.
    \end{flushleft}
    
    \begin{flushleft}
        For the stability, point $(0, 0)$ is \textbf{unstable} because entrance to this equilibrium point is along the line $S = 0$, which over time will make $I$ to decline to 0. This makes sense because the infected individuals cannot exists with the prior absence of susceptible individuals.
        
        On the other hand, point $(\frac{k}{\beta}, \frac{b}{\beta})$ is \textbf{stable} because the trajectory that passed this point will always stay close as $t \to \infty$.

    \end{flushleft}
\end{document}