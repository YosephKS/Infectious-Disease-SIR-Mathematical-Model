\documentclass[../main.tex]{subfiles}

\begin{document}
    \begin{flushleft}
        \textbf{Conclusion}
    \end{flushleft}
    
    \begin{flushleft}
        Thus, based on the phase plane generated by the Python program (\textit{written in Jupyter Notebook}), only the graph that has $\Delta = 0.001$ will converge. Otherwise, other results are representing inaccurate representation of the system of ODE.
    \end{flushleft}
    \begin{flushleft}
        Another interesting thing to point out is that, from the result in Jupyter Notebook, the higher the intiial value is, the "nearer" its trajectory to the equilibrium point $(\frac{k}{\beta}, \frac{b}{\beta})$.
    \end{flushleft}
    \begin{flushleft}
        In addition to the plot of $S-I$ phase plane, another plot, which pictures the evolution of number of susceptible individuals and infected individuals. which shows fluctuation over time, meaning that the numbers of susceptible individuals and infected individuals will take turns in reaching its maximum value and dropping to its minimum value. And, accordingly to the model interpretation, the susceptible individuals must exist initially or have higher number before the infected individuals outnumbered the susceptible ones. It is important to note that similar to the $S-I$ plane, for $\Delta = 0.1$ and $\Delta = 0.01$, the plot have some inaccuracies.
    \end{flushleft}
    
    
\end{document}